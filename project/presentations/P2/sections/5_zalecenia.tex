\section{Zalecenia dotyczące prywatności}

\begin{frame}{Świadomość to podstawa}
\begin{alertblock}{Eksperci są zgodni}
Nasza prywatność w sieci zależy przede wszystkim od:
\begin{itemize}
    \item naszej świadomości,
    \item codziennych nawyków,
    \item poziomu ostrożności.
\end{itemize}
\end{alertblock}
\pause
\begin{exampleblock}{Cel: Ochrona przed wyciekiem danych}
Zadbaj o to, co publikujesz. Twoje dane = Twoje bezpieczeństwo.
\end{exampleblock}
\end{frame}

\begin{frame}{Czego nie udostępniać w internecie?}
\begin{itemize}
    \item Adres e-mail, numer telefonu - spam i phishing
    \item Adres domowy, geolokalizacja - ryzyko śledzenia
    \item Zdjęcia dzieci - ochrona prywatności najmłodszych
    \item Kompromitujące zdjęcia - utrata kontroli
    \item Dokumenty osobiste - kradzież tożsamości
    \item Komentarze, skargi - reputacja, odpowiedzialność
    \item Prywatne rozmowy - korzystaj z szyfrowanych komunikatorów \cite{czegoNieUdostepniac}
\end{itemize}
\end{frame}

\section{Jak się chronić?}

\begin{frame}{Zasady cyfrowej higieny}
\begin{columns}[c]
    \column{0.5\textwidth}
    \begin{itemize}
        \item Aktualizuj oprogramowanie
        \item Zabezpiecz sieć Wi-Fi
        \item Używaj silnych haseł
        \item Włącz uwierzytelnianie 2FA
        \item Zastrzeż swój PESEL
    \end{itemize}
    \column{0.5\textwidth}
    \begin{itemize}
        \item Konfiguruj ustawienia prywatności
        \item Nie klikaj podejrzanych linków
        \item Analizuj usługi „darmowe”
        \item Wyłącz geolokalizację, jeśli niepotrzebna \cite{PROTECTION}
        \item Monitoruj, czy Twoje dane nie wyciekły
    \end{itemize}
\end{columns}
\end{frame}