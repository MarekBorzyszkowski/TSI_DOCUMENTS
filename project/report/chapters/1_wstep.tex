\chapter{WSTĘP I CEL RAPORTU}
\label{chap:introduction}
W dobie powszechnej cyfryzacji coraz więcej aspektów naszego życia przenosi się do świata wirtualnego. 
Codziennie korzystamy z mediów społecznościowych, wyszukiwarek internetowych, aplikacji mobilnych i wielu innych platform, 
nie zawsze zastanawiając się, jakie informacje o sobie udostępniamy i jakie mogą być tego konsekwencje. 
Dane, które pozostawiamy w sieci~-~świadomie lub nieświadomie~-~mogą obejmować zarówno podstawowe informacje, 
takie jak imię i nazwisko, wiek czy miejsce zamieszkania, jak i bardziej wrażliwe dane, takie jak zainteresowania, 
preferencje zakupowe czy historia przeglądania stron. 
Współczesny Internet sprawia, że budowanie profilu użytkownika na podstawie dostępnych informacji jest niezwykle łatwe, 
a same dane mogą być wykorzystywane na różne sposoby przez rozmaite podmioty.

Niniejszy raport ma na celu analizę skali i sposobów udostępniania danych w Internecie. 
W kolejnych rozdziałach omówiona zostanie budowa profilu użytkownika na podstawie podstawowych informacji, 
źródła pozyskiwania danych oraz sposoby ich wykorzystywania przez firmy, instytucje i inne organizacje. 
Szczególna uwaga zostanie poświęcona zagrożeniom wynikającym z nadmiernego udostępniania informacji, takim jak kradzież tożsamości, 
nieuprawnione śledzenie aktywności czy manipulacja preferencjami użytkowników. 
W końcowej części raportu przedstawione zostaną rekomendacje dotyczące ochrony prywatności 
i minimalizacji ryzyka związanego z publikowaniem danych w sieci.
\section{Cel pracy}
Celem niniejszego raportu jest zbadanie, w jakim stopniu i w jaki sposób użytkownicy Internetu udostępniają swoje dane oraz jakie niesie to za sobą konsekwencje. 
Szczególny nacisk zostanie położony na analizę różnych źródeł informacji, mechanizmów zbierania danych oraz podmiotów, które je przetwarzają i wykorzystują.
Poprzez dokładne przeanalizowanie tego procesu możliwe będzie wskazanie zagrożeń związanych z nadmiernym udostępnianiem informacji oraz ocena, 
w jakim stopniu użytkownicy mają nad nim kontrolę.

Drugim, równie istotnym celem raportu, jest przedstawienie praktycznych zaleceń dotyczących ochrony prywatności w Internecie. 
Omówione zostaną sposoby minimalizowania ilości udostępnianych danych, techniki zabezpieczania informacji oraz narzędzia, 
które mogą pomóc w ochronie tożsamości cyfrowej. Ostatecznym efektem pracy będzie zwiększenie świadomości czytelników na temat zagrożeń i sposobów 
zabezpieczenia swoich danych w sieci, co pozwoli im podejmować bardziej świadome decyzje dotyczące własnej prywatności.

