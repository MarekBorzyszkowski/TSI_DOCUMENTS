\chapter{ZAGROŻENIA}
\section{Zagrożenia wynikające z udostępniania danych}

Współczesne organizacje i użytkownicy indywidualni coraz częściej stają się ofiarami zagrożeń wynikających z niekontrolowanego udostępniania danych. Informacje osobiste mogą zostać wykorzystane zarówno przez zewnętrznych atakujących, jak i osoby z wewnątrz firmy — obecnych lub byłych pracowników, kontrahentów, a nawet przypadkowych użytkowników popełniających błędy.

\subsection{Kradzież tożsamości}
Udostępnienie danych osobowych, takich jak imię, nazwisko, PESEL, numer dowodu czy adres e-mail, może skutkować:
\begin{itemize}
\item podszywaniem się pod ofiarę w celu zaciągnięcia pożyczki, otwarcia konta bankowego czy przeprowadzenia transakcji,
\item wykorzystaniem danych w oszustwach ubezpieczeniowych, zdrowotnych lub podatkowych,
\item długoterminowymi problemami finansowymi i prawnymi ofiary.
\end{itemize}

Przykład z życia: W ataku na Marriott w 2020 r. wyciekło ponad 5 mln rekordów gości, zawierających dane kontaktowe, urodziny i numery kont lojalnościowych.

\subsection{Naruszenie prywatności i poufności danych}
Dane, które zostaną ujawnione — nawet nieświadomie — mogą naruszyć prywatność pracowników, klientów czy partnerów biznesowych. Skutki to:
\begin{itemize}
\item utrata kontroli nad własnymi informacjami,
\item możliwość ich dalszego rozpowszechniania,
\item narażenie na nieautoryzowane monitorowanie.
\end{itemize}
Przykład z życia: były pracownik Tesli w 2023 r. ujawnił dane ponad 75 000 pracowników, w tym numery SSN i informacje finansowe.

\subsection{Cyberstalking i nękanie}
Gdy dane kontaktowe lub lokalizacyjne stają się publiczne, rośnie ryzyko:
\begin{itemize}
\item śledzenia i nękania w sieci lub w świecie rzeczywistym,
\item stosowania szantażu lub gróźb,
\item prób kontaktu ze strony osób niepożądanych, np. stalkerów.
\end{itemize}
Przykład z życia: lokalizacja kobiety w Japonii po odbiciu w oku na zdjęciu.

\subsection{Oszustwa socjotechniczne i internetowe}
Ujawnione dane są wykorzystywane do:
\begin{itemize}
\item phishingu (fałszywe e-maile i strony),
\item spear-phishingu (ukierunkowane ataki na konkretną osobę),
\item vishingu (oszustwa telefoniczne),
\item przekonywania ofiar do ujawnienia loginów i haseł.
\end{itemize}

Przykład: w 2020 r. grupa atakujących uzyskała dostęp do systemów Twittera, podszywając się pod pracowników i kradnąc dane do kont znanych osób.

\subsection{Konsekwencje reputacyjne}
Dla osób publicznych, pracowników wysokiego szczebla lub właścicieli firm, ujawnienie danych może prowadzić do:
\begin{itemize}
\item utracenia zaufania klientów i partnerów,
\item negatywnego rozgłosu medialnego
\item trudności w znalezieniu nowej pracy lub rozwijaniu kariery.
\end{itemize}

Przykład: wyciek danych w firmach takich jak Proofpoint, Apple czy Google nadszarpnął ich wizerunek jako liderów w dziedzinie innowacji i bezpieczeństwa.

Przykład: dane z Facebooka w 33\% spraw rozwodowych.

\subsection{Odpowiedzialność prawna}
Udostępnianie danych bez zgody lub ich niewłaściwe zabezpieczenie może skutkować:
\begin{itemize}
\item pozwami cywilnymi ze strony poszkodowanych osób,
\item karami finansowymi za naruszenie przepisów o ochronie danych (np. RODO/GDPR),
\item odpowiedzialnością dyscyplinarną lub karną.
\end{itemize}

Przykład: Microsoft mógłby zostać ukarany grzywną do 20 mln euro za wyciek danych z systemów GitHub, gdyby doszło do naruszenia danych klientów z UE.

\section{Przykłady rzeczywistych incydentów \cite{zagrozeniaFirmy, zagrozeniaLudzie}}
%TODO to będą pewnie jakies zdjęcia
\begin{itemize}
  \item \textbf{David Quintavalle} - błędnie oskarżony na podstawie zdjęcia.
  \item \textbf{Angela Dunnn} - protesty po ujawnieniu adresu.
  \item \textbf{Guy Babcock} - fałszywe oskarżenia o przestępstwa seksualne.
  \item \textbf{Yue Chen} - ustalenie adresów lekarzy do planowania ataku.
\end{itemize}