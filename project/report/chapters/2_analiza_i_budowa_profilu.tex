\chapter{ANALIZA PODSTAWOWYCH DANYCH I BUDOWA PROFILU}

\section{Zakres i charakter danych osobowych w kontekście budowy profilu}

Analizując proces gromadzenia i wykorzystywania informacji, warto najpierw zrozumieć, co w świetle różnorodnych definicji i regulacji uznaje się za \emph{dane osobowe}. W literaturze i dokumentach anglojęzycznych często używa się określenia \textbf{Personally Identifiable Information (PII)}\cite{PII_USDE}, które obejmuje wszelkie dane pozwalające na bezpośrednie lub pośrednie zidentyfikowanie konkretnej osoby. W kontekście ustawodawstwa europejskiego termin ten pokrywa się w znacznej mierze z terminem \textbf{dane osobowe} w rozumieniu \emph{RODO}, a w praktyce oznacza wszelkie informacje związane z osobą fizyczną, które mogą posłużyć do jej zidentyfikowania.

\subsection{Elementy PII i podstawowe dane osobowe}
Według definicji \emph{U.S. Department of Energy}\cite{PII_USDE}, za PII uznaje się:
\begin{itemize}
    \item Dane \emph{jednoznacznie} identyfikujące daną osobę: między innymi imię, nazwisko, numer paszportu, numer PESEL (w kontekście polskim) lub inne unikalne identyfikatory.
    \item Dane, które \emph{pozwalają} na ustalenie tożsamości osoby, nawet jeśli nie ma w nich bezpośrednich identyfikatorów. Może to być data urodzenia, kod pocztowy czy stan cywilny, jeśli te informacje odpowiednio skorelowane z innymi źródłami umożliwiają wskazanie konkretnej osoby.
\end{itemize}

W polskich realiach \textbf{imię i nazwisko} także stanowią dane osobowe, choć w niektórych sytuacjach \emph{same w sobie} mogą nie być wystarczające do jednoznacznej identyfikacji - zależy to od kontekstu. Niemniej jednak, jak podkreślają eksperci\cite{LEXDIGITAL_CZY_IMIE_NAZW_TO_DANE_OS}, imię i nazwisko, w połączeniu z innymi informacjami (takimi jak miejsce zamieszkania czy data urodzenia), zdecydowanie umożliwia powiązanie takich danych z konkretną jednostką.

\subsection{Jakie dane najczęściej ujawniamy w Internecie?}
Według analizy przeprowadzonej na portalu \emph{CyberDefence24}\cite{CYB_DEF_NAJCZĘSTRZE_DANE}, większość internautów regularnie udostępnia:
\begin{itemize}
    \item \textbf{Imię i nazwisko} - choćby na portalach społecznościowych lub forach, gdzie rejestracja wymaga konta opartego na rzeczywistych danych.
    \item \textbf{Adres e-mail} i \textbf{numer telefonu} - używane powszechnie do zakładania kont, newsletterów lub w procesach odzyskiwania haseł.
    \item \textbf{Zdjęcia i nagrania wideo} - w wielu przypadkach z metadanymi (np. lokalizacją GPS), które mogą ujawniać dodatkowe informacje.
    \item \textbf{Informacje o miejscu zamieszkania} - często w postach dotyczących codzienności lub jako część ustawień konta (w tym np. kod pocztowy).
    \item \textbf{Zainteresowania, poglądy, aktywności} - w formie polubień stron, członkostwa w grupach tematycznych czy interakcji w komentarzach.
\end{itemize}

Co istotne, \emph{nie musimy} samodzielnie publikować wszystkich tych danych. Część informacji zostaje ujawniona poprzez aktywność naszych znajomych i bliskich, którzy np. \emph{oznaczają} nas w lokalizacjach lub na zdjęciach, komentują sytuacje z naszego życia prywatnego czy rodzinnego, a także publikują różnego rodzaju \textbf{listy kontaktów}.

\subsection{Konsekwencje łączenia danych z różnych źródeł}
Nawet jeśli pojedyncza dana (np. adres e-mail) nie wydaje się sama w sobie istotna, to można ją bez trudu \emph{zmapować} z kolejnymi rejestrami, w których podawane są imię, nazwisko czy numer telefonu. Stąd już tylko krok do poznania rozbudowanej historii aktywności danej osoby w Internecie. Teoretycznie drobne fakty składają się na \textbf{profil cyfrowy}:
\begin{itemize}
    \item Dokładniejszy wgląd w preferencje, poglądy i preferowane aktywności.
    \item Możliwość wyśledzenia relacji rodzinnych i towarzyskich.
    \item Analiza trendów w zachowaniu i przewidywanie przyszłych aktywności.
\end{itemize}

W poprzednich częściach raportu wskazywaliśmy przypadki re-identyfikacji, w których kluczowe były właśnie tzw. \emph{quasi-identyfikatory} (data urodzenia, miejsce zamieszkania, płeć). Ich połączenie z innymi zasobami - często ogólnodostępnymi - prowadzi do \emph{deanonimizacji} osób, które uważały się za bezpieczne za sprawą samego ograniczenia ilości ujawnianych danych.

\vspace{12pt}
\section{Profilowanie: definicje, zastosowania i zagrożenia}

Wraz ze wzrostem ilości informacji dostępnych w Internecie, coraz więcej podmiotów, zarówno komercyjnych, jak i publicznych, decyduje się na \textbf{profilowanie danych osobowych}. Jak wskazuje opracowanie PARP\cite{POIR_PROFILOWANIE}, profilowanie to zautomatyzowany proces przetwarzania danych osobowych, służący do oceny określonych czynników dotyczących osoby fizycznej, w szczególności do analizy lub prognozy aspektów związanych z wynikami w pracy, sytuacją ekonomiczną, zdrowiem, preferencjami osobistymi czy zainteresowaniami.

\subsection{Techniki profilowania i sposoby wykorzystania}
Profilowanie jest możliwe przede wszystkim dzięki:
\begin{itemize}
    \item \textbf{Zaawansowanej analityce danych (Data Analytics)} - wykorzystuje się algorytmy uczenia maszynowego i sztucznej inteligencji, które na podstawie dużych zbiorów informacji są w stanie kategoryzować użytkowników i przewidywać ich zachowania.
    \item \textbf{Danym z różnych platform} - integracja danych z serwisów społecznościowych, systemów bankowych, firm ubezpieczeniowych czy sklepów internetowych pozwala na bogate wnioski dotyczące preferencji i ryzyka.
    \item \textbf{Plikom cookies i innym formom śledzenia} - znaczniki i identyfikatory internetowe (np. adres IP, identyfikatory urządzeń mobilnych) umożliwiają śledzenie aktywności na wielu stronach i aplikacjach, budując pełen obraz nawyków użytkowników.
\end{itemize}

W efekcie firmy mogą np. \emph{personalizować} oferty handlowe, reklamy, rekomendacje produktowe czy decyzje kredytowe. Z jednej strony zwiększa to wygodę użytkownika (propozycja usług najlepiej pasujących do jego potrzeb), z drugiej - \emph{rodzi wiele pytań} o kontrolę nad tym, jakie informacje i w jaki sposób są wykorzystywane.

\subsection{Zagrożenia wynikające z profilowania}
Choć profilowanie niesie potencjalne korzyści, wiąże się też z licznymi zagrożeniami:
\begin{enumerate}
    \item \textbf{Brak transparentności} - użytkownik często nie ma świadomości, że jest profilowany ani w jakim zakresie dane na jego temat są gromadzone.
    \item \textbf{Ryzyko dyskryminacji} - przy przyznawaniu kredytów, składek ubezpieczeniowych czy ofert pracy systemy oparte na algorytmach mogą \emph{wykluczać} określone grupy osób.
    \item \textbf{Nieodwracalna identyfikacja} - gdy dane raz zostaną połączone z tożsamością użytkownika, trudno przywrócić wcześniejszy stan anonimowości, zwłaszcza jeśli informacje zostały skopiowane do wielu niezależnych baz.
    \item \textbf{Luka w ochronie prywatności} - różne systemy prawne (np. w USA, w Unii Europejskiej czy w innych regionach) w odmienny sposób regulują zasady przetwarzania danych. W rezultacie przechwycenie danych z jednego kraju i przetworzenie ich w innym może \emph{osłabiać} gwarancje prywatności.
\end{enumerate}

W kontekście przepisów RODO (Rozporządzenie Ogólne o Ochronie Danych Osobowych), każda osoba ma prawo do informacji o profilowaniu oraz, w pewnych przypadkach, do \emph{wniesienia sprzeciwu} wobec takiego przetwarzania. Rzeczywiste zastosowanie tych uprawnień bywa jednak utrudnione, zwłaszcza gdy zakres i sposób pozyskiwania danych nie jest przejrzysty.

\subsection{Implikacje dla budowania profilu}
Mechanizmy profilowania znacząco przyspieszają proces \emph{budowania zaawansowanego profilu}, o czym traktuje niniejszy rozdział raportu. Systemy oparte na uczeniu maszynowym są w stanie:
\begin{itemize}
    \item \textbf{Powiązać różne źródła} danych (np. z mediów społecznościowych, rejestrów genealogicznych, analiz geolokalizacyjnych) w jeden spójny model.
    \item \textbf{Wnioskować} o cechach i zachowaniach, które nie zostały bezpośrednio udostępnione przez użytkownika (np. stan zdrowia, skłonność do podróży, zainteresowanie konkretną dziedziną).
    \item \textbf{Generować prognozy i klasyfikacje}, które nie zawsze są zgodne z rzeczywistością, ale w wystarczająco wielu przypadkach pozwalają na skuteczne podejmowanie decyzji handlowych, politycznych czy społecznych.
\end{itemize}

\noindent W połączeniu z \textbf{re-identyfikacją}, omawianą w poprzednich sekcjach, profilowanie tworzy środowisko, w którym pojedynczy użytkownik traci kontrolę nad własnym wizerunkiem i prywatnością w przestrzeni cyfrowej. Z tego względu rośnie znaczenie edukacji społecznej na temat świadomości cyfrowej oraz \emph{podstawowych praktyk} związanych z ochroną danych.

\vspace{12pt}
\paragraph{Wnioski i perspektywy}
\noindent Omawiane w tej sekcji zjawiska ilustrują, jak budowanie profilu użytkownika \emph{wykracza} daleko poza analizę podstawowych danych osobowych. Dodatkowe techniki i narzędzia (takie jak profile behawioralne, narzędzia Big Data czy rozpoznawanie wzorców w aktywnościach online) \textbf{poszerzają} zakres i dokładność takiej analizy. Jednocześnie zmienia się status prawny i społeczny użytkowników, którzy muszą mierzyć się z \emph{możliwą dyskryminacją} lub \emph{utrudnieniem} dostępu do usług, jeśli wynik profilowania okaże się dla nich niekorzystny.

W kolejnych rozdziałach przyjrzymy się dokładniej mechanizmom i strategiom minimalizowania ryzyka związanego z profilowaniem oraz re-identyfikacją. Zastanowimy się też, jak w praktyce \emph{ograniczyć} nadmierne udostępnianie danych, aby nasza aktywność w Internecie nie stała się źródłem niekontrolowanego przepływu informacji o naszej tożsamości, zachowaniach i preferencjach.

\section{Wykorzystanie OSINT w analizie podstawowych danych}

\noindent
Istotnym elementem pozyskiwania informacji, pozwalającym na \emph{ulepszenie} profilu osoby już na wczesnym etapie analizy, jest \textbf{OSINT (Open Source Intelligence)}. Polega on na wykorzystywaniu otwartych i publicznie dostępnych źródeł danych, co w znacznym stopniu ułatwia skorelowanie nawet \emph{podstawowych faktów} (np. daty urodzenia, miejsca zamieszkania, informacji genealogicznych) z dodatkowymi szczegółami na temat konkretnej jednostki.

\subsection{Kluczowe zasoby i narzędzia}
Istnieje szereg zróżnicowanych repozytoriów i narzędzi wspierających OSINT, z czego najpopularniejsze to:
\begin{itemize}
    \item \emph{OSINT Framework}\cite{OSINT_FRAMEWORK} - interaktywna „mapa” serwisów i aplikacji służących do wyszukiwania osób, adresów e-mail, numerów telefonów czy aktywności w mediach społecznościowych.
    \item \emph{IntelTechniques}\cite{INTELTECHNIQUES} - obszerna kolekcja linków i skryptów, ułatwiających przeszukiwanie konkretnych platform i rejestrów pod kątem danych o osobie.
\end{itemize}

\subsection{Przykłady zastosowania OSINT w budowaniu profilu}
W kontekście \textbf{analizy podstawowych danych i tworzenia profilu} jednostki, OSINT wspomaga:
\begin{itemize}
    \item \textbf{Wyszukiwanie danych kontaktowych i danych demograficznych} (m.in. imię, nazwisko, wiek, płeć, miejsce zamieszkania), które często można odnaleźć na publicznych profilach w mediach społecznościowych lub w rejestrach online.
    \item \textbf{Powiązywanie istniejących informacji} - np. data i miejsce urodzenia \emph{zestawione} z rejestrami genealogicznymi, co pozwala szybciej zidentyfikować członków rodziny oraz kluczowe fakty biograficzne danej osoby.
    \item \textbf{Pogłębianie wiedzy} o aktywnościach internetowych (dawne posty, wzmianki w forach, archiwach stron), co umożliwia uzupełnienie profilu o dodatkowe szczegóły, takie jak zainteresowania lub przybliżony stan majątkowy.
\end{itemize}

\subsection{Znaczenie ogólnodostępnych źródeł}
Wspomniane narzędzia bazują głównie na \textbf{powszechnie dostępnych danych}, do których należą m.in.:
\begin{itemize}
    \item \textbf{Serwisy genealogiczne} (np. FamilySearch, WikiTree), zawierające dane o dacie i miejscu urodzenia czy historii rodziny.
    \item \textbf{Rejestry i bazy urzędowe} (krajowe i międzynarodowe) - od podstawowych statystyk demograficznych po niektóre listy adresowe.
    \item \textbf{Platformy społecznościowe}, gdzie użytkownicy nierzadko sami dzielą się informacjami, \emph{ułatwiając} tym samym ustalenie ich tożsamości oraz najważniejszych cech profilu.
\end{itemize}

\subsection{Znaczenie OSINT na etapie „podstawowych danych”}
W ramach \textbf{pierwszego etapu} analizy, gdy punktem wyjścia są \emph{proste} informacje (wiek, płeć, region zamieszkania), OSINT pomaga w:
\begin{itemize}
    \item \textbf{Weryfikacji autentyczności} - potwierdzeniu, czy dane elementy (np. data urodzenia) są spójne z innymi źródłami.
    \item \textbf{Szybszym zawężeniu obszaru poszukiwań} - przeszukiwanie publicznych zasobów z użyciem narzędzi OSINT pozwala od razu odrzucić dane niewłaściwe i skupić się na informacjach naprawdę użytecznych w budowie profilu.
    \item \textbf{Korelowaniu różnych identyfikatorów} - przykładowo, adres e-mail zarejestrowany w jednym serwisie może być powiązany z kontem w innej platformie, co otwiera dostęp do kolejnych danych.
\end{itemize}

\subsection{Podsumowanie}
OSINT stanowi \textbf{skuteczne wsparcie} w procesie analizy podstawowych danych i \textbf{budowania profilu} osoby, uwidaczniając, jak wiele szczegółów można ustalić \emph{bez} zastosowania wyspecjalizowanych, zamkniętych baz czy usług komercyjnych. Otwarta struktura Internetu oraz rosnąca liczba publicznie dostępnych rejestrów sprawiają, że umiejętne korzystanie z OSINT stanowi nieodłączny element każdego przedsięwzięcia, w którym celem jest zebranie jak najbardziej szczegółowych i zweryfikowanych informacji na temat danej osoby.

\section{Przykład re-identyfikacji - jak pozornie anonimowe dane mogą ujawniać tożsamość}

W ramach rozdziału warto przyjrzeć się przykładowi opisującemu, jak dane uznawane za anonimowe mogą zostać użyte do ponownej identyfikacji konkretnych osób. Przypadek opisany przez Adama Tannera w magazynie „Forbes” \cite{DNA_LEAK}

odnosi się do sytuacji, w której badacz z Uniwersytetu Harvarda połączył próbki DNA rzekomo „anonimowych” ochotników z publicznie dostępnymi danymi genealogicznymi. Dzięki temu zdołał przypisać je do konkretnych osób.

\subsection{Re-identyfikacja na podstawie danych podstawowych}

W omawianym przykładzie kluczową rolę odgrywały wyłącznie \textit{podstawowe informacje}, takie jak:
\begin{itemize}
    \item Dane demograficzne (np. rok urodzenia, płeć, stan cywilny).
    \item Częściowe dane genealogiczne (np. nazwiska przodków, zarys drzewa rodowego).
    \item Ogólnodostępne rejestry i bazy, gdzie można zestawić powyższe informacje (np. rejestry ludności, dokumenty archiwalne).
\end{itemize}

Okazało się, że nawet przy anonimowych próbkach DNA można użyć \textbf{drobnych, publicznie dostępnych faktów} do precyzyjnego ustalenia tożsamości, jeśli tylko połączy się ze sobą odpowiednie elementy informacyjne pochodzące z wielu źródeł. Analiza ta nie wymagała specjalistycznych czy drogich rozwiązań; wystarczyły ogólnodostępne bazy genealogiczne oraz archiwa, które może przeszukiwać każdy internauta.

\paragraph{Znaczenie tzw. \emph{quasi-identyfikatorów}}
Kluczowe w procesie re-identyfikacji okazują się dane, które \emph{nie zawierają} jednoznacznych oznaczeń personalnych (np. imię i nazwisko wprost), ale w połączeniu z innymi źródłami tworzą wyraźny profil jednostki. Takie dane często nazywa się \textbf{quasi-identyfikatorami}. Należą do nich m.in.:
\begin{itemize}
    \item data urodzenia (w szczególności dzień i miesiąc),
    \item kod pocztowy lub nazwa miejscowości,
    \item płeć i wiek,
    \item stan cywilny czy wykształcenie.
\end{itemize}
Choć z osobna nie stanowią kompletnego zestawu, to kiedy zestawi się je z innymi publicznymi rejestrami (np. spisy powszechne czy listy mieszkańców), zwykle szybko można zawęzić krąg osób do kilku lub wręcz jednej.

\paragraph{Przykłady łączenia danych}
Praktyka pokazuje, że re-identyfikacja bywa możliwa nawet w sytuacjach, gdy dane \emph{pozornie} zostały zanonimizowane. Oprócz badań genetycznych, podobne zagrożenia występują w innych dziedzinach:
\begin{itemize}
    \item \textbf{Informacje medyczne:} w przeszłości dowiedziono, że dane medyczne (np. rekordy ze szpitali) można powiązać z osobami, jeśli połączy się je z danymi demograficznymi z rejestrów urzędowych.
    \item \textbf{Logi aktywności internetowej:} w przypadku wycieków logów wyszukiwarki internetowej, nawet częściowo zanonimizowane identyfikatory internautów okazywały się wystarczające do wydobycia szczegółów o ich życiu prywatnym (np. miejsca zamieszkania, zainteresowań, problemów zdrowotnych).
    \item \textbf{Bazy genealogiczne:} gdy ktoś upublicznia fragment drzewa rodzinnego (nazwiska przodków, daty narodzin i zgonów), badacz może \emph{dosztukować} brakujące elementy, korzystając z rejestrów lub innych źródeł, w efekcie dopasowując profil do konkretnej osoby.
\end{itemize}

\paragraph{Dynamika zmian w dostępnych źródłach}
Nie należy zapominać, że dostępność danych w Internecie \textbf{nieustannie się poszerza}. Dzięki digitalizacji i masowemu udostępnianiu zasobów (np. projektom bibliotek cyfrowych czy otwieraniu kolejnych baz urzędowych), dzisiejsza „bezpieczna” porcja informacji może jutro okazać się niewystarczająca do zachowania prywatności. Wraz z pojawianiem się nowych zbiorów - na przykład szczegółowych map geolokalizacyjnych, kolejnych edycji spisów powszechnych czy społecznościowych inicjatyw genealogicznych - \emph{również historyczne dane zyskują nowy kontekst} i mogą prowadzić do ujawnienia wrażliwych szczegółów.

\paragraph{Podsumowanie zagadnienia re-identyfikacji}
Zjawisko re-identyfikacji opartej na podstawowych danych unaocznia, że proces anonimizacji powinien uwzględniać coraz to nowsze i łatwiej dostępne źródła informacji. Jeśli nawet kilka lat temu uznano jakiś zbiór za bezpieczny (bo odarte z niego dane uznano za „nieszkodliwe”), to w obecnej rzeczywistości, gdzie poszczególne rejestry można swobodnie łączyć, \textbf{tylko szczególnie rygorystyczne metody anonimizacji} dają realną gwarancję ochrony prywatności. W praktyce zaś większość danych „anonimowych” może zostać \emph{deanonimizowana}, o ile dysponuje się wystarczającymi zasobami i wiedzą do ich korelowania z ogólnodostępnymi bazami.

\subsection{Rola otwartych i powszechnie dostępnych źródeł}

Zasadniczy problem ujawniony w tej historii polega na tym, że spora część naszych danych znajduje się w \textbf{otwartych i łatwo dostępnych} zasobach internetu. Dotyczy to m.in.:

\begin{enumerate}
    \item \textbf{Publicznych baz rządowych i organizacyjnych}:
    \begin{itemize}
        \item W Polsce są to choćby \url{https://dane.gov.pl/} czy \url{https://bdl.stat.gov.pl/} (Bank Danych Lokalnych GUS).
        \item W USA bogatym źródłem statystyk jest \url{https://data.census.gov/}, a na poziomie europejskim - \url{https://data.europa.eu/}.
        \item Informacje o zameldowaniach, urodzeniach czy zgonach znajdują się w różnorodnych spisach i rejestrach regionalnych.
    \end{itemize}

    \item \textbf{Serwisów genealogicznych i historycznych}:
    \begin{itemize}
        \item Popularne platformy (np. \url{https://www.familysearch.org/}, \url{https://www.wikitree.com/}) gromadzą nazwiska, daty, a czasem nawet dokumenty rodzinne, co umożliwia odtworzenie fragmentów drzew genealogicznych.
        \item W niektórych krajach udostępniane są rejestry małżeństw, akt urodzeń i zgonów, które - choć ograniczone prawem - często dają się przeglądać bez większych formalności.
    \end{itemize}

    \item \textbf{Mediów społecznościowych i forów tematycznych}:
    \begin{itemize}
        \item Nawet jeśli użytkownicy nie udostępniają publicznie wszystkich danych, to często dzielą się szczegółami (np. datą urodzenia, miejscem zamieszkania, relacjami rodzinnymi) w otwartych grupach czy w komentarzach.
        \item Analiza np. zdjęć może zdradzić lokalizację (dzięki metadanym GPS lub charakterystycznym elementom w tle).
    \end{itemize}
\end{enumerate}

Dzięki wymienionym zasobom, osoba mająca wystarczającą motywację i podstawową wiedzę o metodach wyszukiwania jest w stanie zebrać rozmaite informacje o konkretnej jednostce. Początkowo są to drobne fakty (wiek, miejsce pochodzenia), jednak po złożeniu wszystkich elementów w całość otrzymujemy \textbf{szczegółowy profil} konkretnej osoby.

\subsection{Konsekwencje dla prywatności i bezpieczeństwa}

Opisany przykład wskazuje, że \textbf{anonimizacja} danych to często proces wyłącznie pozorny, zwłaszcza gdy nie uwzględnia się rosnącej liczby publicznie dostępnych zbiorów, z którymi można je zestawiać. Dla ochrony prywatności istotne jest, aby:

\begin{enumerate}
    \item \textbf{Świadomie udostępniać informacje}: Zastanowić się, czy publikacja konkretnego faktu (np. dokładnej daty urodzenia) jest faktycznie niezbędna w danym miejscu (profil społecznościowy, forum internetowe, ankieta itp.).
    \item \textbf{Rozpoznawać ryzyko łączenia danych}: Nawet dane wrażliwe (np. zdrowotne czy genetyczne) mogą zostać dopasowane do osoby, jeśli w innej bazie lub w mediach społecznościowych widnieje choć część informacji „uzupełniających” (data urodzenia, nazwisko panieńskie matki, nazwa miejscowości).
    \item \textbf{Szukać mechanizmów ochronnych}: Sprawdzać, czy wybrana platforma oferuje wystarczającą kontrolę nad prywatnością i przetwarzaniem danych. Zgłębiać też przepisy takie jak RODO (w Unii Europejskiej) czy HIPAA (w Stanach Zjednoczonych) - regulacje te ustalają m.in. obowiązki podmiotów przetwarzających dane.
\end{enumerate}

\subsection{Podsumowanie: wzór dla pozostałych analiz}

Omówiony przypadek z artykułu z „Forbes” ukazuje zjawisko, które dotyczy nie tylko badań genetycznych, ale praktycznie każdej sytuacji, w której istnieje potencjał do \textbf{połączenia rzekomo anonimowych informacji z innymi zbiorami}. To właśnie w taki sposób, krok po kroku, tworzy się i uszczegóławia profil danej osoby.

