\chapter{INNE PRZYDATNE KONSTRUKCJE}
\section{Symbole otoczone kółkiem}

Latex pozwala na tworzenie własnych znaków i symboli. W mojej rozprawie doktorskiej potrzebowałem zestawu symboli zamkniętych w okrąg, np.\ \mycircalign{+}, \mycircalign{-}, \mycircalign{k}. Za pomocą pakietu \textbf{tikz} zdefiniowałem dwa makra: \textbf{mycircalign} oraz \textbf{mycirc}. Pierwsze z nich otacza symbol kółkiem i wyrównuje powstały obrazek tak, by jego środek pokrywał się ze środkiem sąsiadujących znaków. Drugi układa elementy na linii tekstu, tak że spody znaków sąsiadujących i powstałego obrazka są wyrównane. Jak każde makro, również i te można używać zarówno w tekście jak i w środowisku matematycznym, tabelkach itp. (Równanie~\ref{eq:makro}).

\begin{equation}
  \mathcal{O}_3 = \mathcal{O}_1 \mycircalign{+} \mathcal{O}_2
  \label{eq:makro}
\end{equation}

\section{Tymczasowa zmiana rozmiaru strony}

Czasami zachodzi konieczność chwilowej zmiany rozmiaru strony, by np.\ udało się zmieścić jedną dodatkową linijkę tekstu. Możemy to wykonać za pomocą polecenia \textbf{enlargethispage\{\}}, gdzie jako parametr podajemy rozmiar oraz jednostkę. Należy pamiętać, że polecenie to musi zostać wydane odpowiednio wcześniej, by Latex zdążył zastosować nowy rozmiar strony. Najlepiej by te polecenie było bezpośrednio przed nową stroną, jednak zazwyczaj jest to kwestia poeksperymentowania. Wartość przekazana jako parametr może być także ujemna. Przykładowo strona zawierająca rozdział~\ref{listingi} została w niniejszym dokumencie pomniejszona o 5 cm wymuszając przeniesienie początku rozdziału~\ref{algorytmy} na następną stronę.

Strona zawierająca rozdział~\ref{rysunki} powiększono o~20~punktów, co pozwoliło uniknąć samotnej, pojedynczej linii tekstu kończącego akapit (tzw.\ wdowy) na następnej stronie. Takie drobne modyfikacje rozmiaru strony zazwyczaj są niezauważalne dla czytelnika a poprawiają ogólny układ dokumentu.

\section{Wpisy bibliograficzne}

Wpisy bibliograficzne przechowujemy w odrębnym pliku z rozszerzeniem\ .bib. Przykładowy plik został dołączony do tego dokumentu. Do pliku takiego należy dodawać odpowiedni sformatowane wpisy. Latex automatycznie posortuje je po nazwiskach autorów oraz do finalnego dokumentu dołączy tylko te wpisy,które posiadają odwołania w tekście! Można więc stworzyć kompletną bazę publikacji, a Latex użyje tylko to co potrzeba. Dodatkowo, dzięki użyciu pakietu \textbf{natbib} z parametrem \textbf{sort} (patrz preambuła dokumentu), numerki w odwołaniach również zostaną posortowane, niezależnie od kolejności podania odwołań. Przykład podano w rozdziale~\ref{podzial}.