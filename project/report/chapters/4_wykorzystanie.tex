\chapter{WYKORZYSTANIE DANYCH PRZEZ RÓŻNE PODMIOTY}

\noindent
Współcześnie informacje gromadzone na temat użytkowników Internetu stały się \textbf{strategicznym zasobem} w wielu branżach. Dane te mogą być wykorzystywane zarówno w legalnych i powszechnie akceptowanych procesach biznesowych, jak i w działaniach sprzecznych z prawem oraz zasadami etyki. W niniejszym rozdziale omówiono główne kierunki wykorzystania danych przez różnorodne podmioty, ze szczególnym uwzględnieniem sektora reklamowego, firm rekrutacyjnych oraz środowiska cyberprzestępców.

\vspace{1em}

\section{Firmy reklamowe i sektor marketingu}

\subsection{Personalizacja i targetowanie reklam}
Dynamiczny rozwój platform cyfrowych umożliwił firmom reklamowym gromadzenie i przetwarzanie ogromnych ilości danych, takich jak:
\begin{itemize}
    \item \emph{historia przeglądania stron internetowych},
    \item \emph{aktywność w mediach społecznościowych},
    \item \emph{dane transakcyjne} (np. preferencje zakupowe, częstotliwość dokonywania zakupów online).
\end{itemize}
Jak podkreśla raport \emph{Behind the One-Way Mirror} opublikowany przez EFF\cite{EFF}, tak szeroki zakres informacji pozwala na \textbf{targetowanie behawioralne}, w którym reklamy są dopasowywane do konkretnych grup odbiorców, uwzględniając ich indywidualne cechy i preferencje.

W praktyce oznacza to, że użytkownikowi, który często wyszukuje informacje o zdrowym stylu życia, mogą się wyświetlać reklamy związane z dietą i aktywnością fizyczną. Natomiast osoby zainteresowane planowaniem podróży będą otrzymywać propozycje ofert turystycznych. Mechanizm ten, choć zwiększa \emph{trafność} komunikatów marketingowych, jednocześnie - jak wskazują liczne analizy rynkowe - \textbf{podnosi} skuteczność kampanii reklamowych i skłania użytkowników do częstszych interakcji z treściami sponsorowanymi.

\subsection{Wykorzystanie Big Data w analizie zachowań konsumenckich}
Zgodnie z danymi statystycznymi przywoływanymi w raporcie \emph{Digital 2025 Global Overview Report}\cite{DIGITAL_GLOBAL}, liczba użytkowników Internetu stale rośnie, a firmy reklamowe i agencje marketingowe mogą dzięki temu gromadzić coraz bardziej rozbudowane zasoby Big Data. Informacje o ścieżkach zakupowych czy upodobaniach pozwalają:
\begin{itemize}
    \item \textbf{segmentować} odbiorców według wielu kryteriów (np. styl życia, wiek, dochód),
    \item \textbf{przewidywać} zachowania (np. kto najchętniej dokona zakupu w danej kategorii produktów),
    \item \textbf{monitorować} skuteczność kampanii w czasie rzeczywistym.
\end{itemize}

Takie dane służą również do tworzenia \emph{dynamiki cenowej} (tzw. \emph{price discrimination}), gdzie użytkownicy otrzymują zindywidualizowane oferty w zależności od historii wyszukiwań i wcześniejszych interakcji z reklamami. Dzięki temu możliwe jest zwiększanie zysków firm, ale pojawiają się też wątpliwości etyczne co do stopnia ingerencji w prywatność konsumentów.

\subsection{Nowe formaty reklamowe i ekosystem wymiany danych}
Raporty dotyczące rynku cyfrowego wskazują, że w obszarze marketingu zachodzi \emph{przesunięcie} z tradycyjnych form reklamy (banery, wyskakujące okienka) w stronę treści natywnych oraz tzw. \textbf{mikro-targetowania}. W tym procesie kluczową rolę pełnią:
\begin{itemize}
    \item \textbf{wielkie platformy internetowe} (np. Google, Facebook, TikTok), które gromadzą i scalają dane o aktywnościach użytkowników,
    \item \textbf{agencje analityczne} oferujące narzędzia do profilowania i śledzenia zachowań online,
    \item \textbf{marketerzy i działy komunikacji} w firmach, które poszukują maksymalnego zwrotu z inwestycji w kampanie reklamowe.
\end{itemize}
W efekcie cały ekosystem \emph{nieustannie} się rozrasta, a dane o preferencjach konsumenckich krążą między różnymi platformami i stają się przedmiotem intensywnej wymiany.

\vspace{1em}

\section{Branża HR i rekruterzy}

\subsection{Analiza śladów w Internecie}
W rekrutacji zawodowej dane osobowe oraz aktywność kandydatów w sieci mogą przesądzać o \emph{ocenie} ich przydatności na określone stanowiska. Jak wynika z lokalnych analiz przywoływanych w \emph{Digital 2025 Local Country Headlines}\cite{DIGITAL_LOCAL}, wzrasta liczba przedsiębiorstw regularnie korzystających z portali zawodowych i mediów społecznościowych w procesach HR. Rekruterzy:
\begin{itemize}
    \item sprawdzają \textbf{wiarygodność} profili kandydatów,
    \item weryfikują doświadczenie zawodowe na podstawie informacji publicznych,
    \item analizują \textbf{styl komunikacji} i \textbf{zainteresowania}, co może mieć znaczenie w ocenie dopasowania do danej kultury organizacyjnej.
\end{itemize}

Dodatkowo rekruterzy często zwracają uwagę na tak zwany \emph{e-wizerunek} kandydata: sposób, w jaki wyraża się on w sieci, jakie treści publikuje i z kim wchodzi w interakcje. Nawet informacje, które kandydat uznaje za nieistotne lub prywatne, mogą rzutować na finalną decyzję o zatrudnieniu.

\subsection{Weryfikacja ścieżek kariery i referencji}
Duże korporacje i agencje rekrutacyjne coraz częściej sięgają po zaawansowane narzędzia analityczne, aby automatycznie oceniać \emph{historię zatrudnienia} kandydata. Dane z portali zawodowych (np. LinkedIn) konfrontuje się z rejestrami branżowymi, co pozwala:
\begin{itemize}
    \item wykryć \emph{nieścisłości} w CV,
    \item zweryfikować \emph{rzetelność} deklarowanych umiejętności,
    \item ustalić ewentualne \emph{powiązania} z kluczowymi organizacjami.
\end{itemize}
Choć taka praktyka bywa postrzegana jako kontrowersyjna w kontekście prywatności, wiele firm utrzymuje, że \emph{szersza analiza} danych usprawnia proces rekrutacji oraz ogranicza ryzyko zatrudnienia osób niedopasowanych do zespołu bądź wprowadzających pracodawcę w błąd.

\vspace{1em}

\section{Cyberprzestępcy i hakerzy}

\subsection{Phishing i spear phishing}
Według \emph{ENISA Threat Landscape 2022}\cite{ENISA} cyberprzestępcy, korzystając z wykradzionych lub publicznie dostępnych informacji, przeprowadzają \textbf{phishing} na skalę masową, a także \textbf{spear phishing} - spersonalizowane kampanie wymierzone w konkretne osoby lub grupy. Mechanizm ataku obejmuje:
\begin{enumerate}
    \item Zbieranie łatwo dostępnych danych o ofierze (stanowisko, profil zawodowy, kontakty).
    \item Przygotowanie \emph{wiarygodnej} wiadomości e-mail lub wiadomości w mediach społecznościowych, często stylizowanej na korespondencję od zaufanego nadawcy.
    \item Skłonienie ofiary do \textbf{kliknięcia w złośliwy link} lub \textbf{otwarcia załącznika} zawierającego malware.
\end{enumerate}
Jak zauważa \emph{APWG Trends Report Q4 2022}\cite{PHISHING_REPORT}, wzrastająca skuteczność inżynierii społecznej sprawia, że phishing jest wciąż jednym z \textbf{najefektywniejszych} sposobów na przejęcie wrażliwych danych.

\subsection{Handel danymi i kradzież tożsamości}
Z raportu \emph{CERT Polska 2023}\cite{CERT_REPORT} wynika, że istotnym problemem jest \textbf{handel skradzionymi danymi}, obejmującymi loginy, hasła, dane osobowe oraz informacje finansowe. Cyberprzestępcy, wchodząc w posiadanie baz danych użytkowników:
\begin{itemize}
    \item dokonują \emph{kradzieży tożsamości} (np. zakładając kredyty w imieniu poszkodowanych),
    \item przejmują \emph{konta w mediach społecznościowych} czy skrzynki e-mail,
    \item odsprzedają dane na \emph{czarnym rynku} (tzw. \textbf{darknet markets}), co dalej napędza rozwój kolejnych ataków.
\end{itemize}
Przestępcy mogą również wykorzystywać dane w formule \textbf{double extortion}, żądając okupu nie tylko za \emph{odszyfrowanie} zasobów, lecz także za \emph{nieujawnienie} przechwyconych informacji.

\subsection{Cybergangi i ataki łańcuchowe}
W ostatnich latach coraz częściej obserwuje się zjawisko \emph{cybergangów}, które profesjonalizują swoje działania, tworząc \textbf{wielopoziomowe łańcuchy} ataków. Jak podkreśla \emph{ENISA Threat Landscape 2022}, procesy te obejmują m.in.:
\begin{itemize}
    \item \textbf{rekonesans} (zbieranie danych o potencjalnych celach z publicznych źródeł),
    \item \textbf{fazę infiltracji} (phishing, spear phishing, przejęcie kont uprzywilejowanych),
    \item \textbf{rozprzestrzenienie} w infrastrukturze ofiary (instalacja złośliwych narzędzi, eskalacja uprawnień).
\end{itemize}
Dane wykradzione w tych kolejnych etapach mają \emph{wartość} zarówno dla samej grupy przestępczej, jak i w szerszym obiegu handlu informacjami.

\vspace{1em}

\section{Brokerzy danych i inni pośrednicy}

\subsection{Koncepcja brokera danych}
Poza bezpośrednimi użytkownikami informacji (firmami reklamowymi, rekruterami czy cyberprzestępcami), na rynku funkcjonuje \textbf{segment brokerów danych}. Są to podmioty specjalizujące się w:
\begin{itemize}
    \item \emph{gromadzeniu} danych z wielu źródeł (zarówno publicznych, jak i komercyjnych),
    \item \emph{scalaniu} ich w rozbudowane profile zawierające szereg informacji demograficznych, behawioralnych czy finansowych,
    \item \emph{odsprzedaży} tych baz firmom bądź organizacjom poszukującym kompleksowych zestawień na temat użytkowników bądź klientów.
\end{itemize}
Mechanizm ten opisują w swoich analizach zarówno \emph{Behind the One-Way Mirror} (EFF), jak i \emph{Digital 2025 Global Overview Report}, wskazując, że obrót danymi w ramach usług brokerskich jest często \emph{niewidoczny} dla przeciętnego internauty.

\subsection{Wykorzystanie danych przez brokerów w analizie rynkowej}
Firmy zajmujące się \textbf{marketingiem}, \textbf{ubezpieczeniami} czy \textbf{rekrutacją} często kupują zbiory danych od brokerów, chcąc uzyskać:
\begin{itemize}
    \item \emph{rozszerzony zestaw informacji} (np. z różnych segmentów działalności użytkownika),
    \item \emph{dane uzupełniające} pozwalające trafniej ocenić kandydata bądź klienta (finanse, historia zakupowa, styl życia),
    \item \emph{opracowane rankingi} i predykcje (np. scoring kredytowy, analiza ryzyka).
\end{itemize}
Zyskiem dla brokerów jest rola pośrednika, który \textbf{konsoliduje} i \textbf{wzbogaca} dane, następnie oferując je zainteresowanym stronom. W rezultacie powstają wyjątkowo \emph{szczegółowe} profile użytkowników, obejmujące dane z rozmaitych etapów ich życia w sieci.

\vspace{1em}

\section{Inne podmioty i obszary wykorzystania}

\subsection{Partnerzy biznesowi i organizacje międzynarodowe}
Wiele międzynarodowych instytucji finansowych czy organizacji gospodarczych podejmuje starania w kierunku \emph{standaryzacji} oraz \emph{wymiany} danych o użytkownikach. Celem jest:
\begin{itemize}
    \item \textbf{prognozowanie} trendów rynkowych w skali globalnej,
    \item \textbf{dostosowanie} usług finansowych do lokalnych potrzeb (np. mikrokredyty w regionach rozwijających się),
    \item \textbf{monitorowanie} zmian w zachowaniach konsumenckich w różnych państwach.
\end{itemize}
Tego rodzaju inicjatywy nie są wprost zorientowane na dane ściśle \emph{personalne} - często chodzi o \emph{zagregowane} statystyki. Niemniej jednak, skala i szczegółowość tych raportów nierzadko pozwala na bardzo wnikliwe wnioski o grupach użytkowników.

\subsection{Sektor naukowo-badawczy}
Część ośrodków naukowych i think-tanków wykorzystuje dane pochodzące z sieci w projektach badawczych, skupiając się na analizach statystycznych czy tworzeniu narzędzi służących zwiększeniu bezpieczeństwa. Jak wskazują analizy cytowane w \emph{ENISA Threat Landscape 2022}, rosnące znaczenie cyberbezpieczeństwa motywuje do gromadzenia i badania \emph{zagregowanych} informacji o incydentach, lukach w zabezpieczeniach czy metodach ataku. Choć dane te są \emph{zazwyczaj} zanonimizowane, sam fakt, że można je precyzyjnie opisywać i analizować, świadczy o bogactwie informacji pojawiających się w przestrzeni publicznej.

\vspace{1em}

\section{Przykłady synergii między podmiotami}

\subsection{Wspólne cele i nieoczywiste współprace}
Warto podkreślić, iż \textbf{różne} rodzaje organizacji (np. korporacje reklamowe, brokerzy danych, a nawet cyberprzestępcy) niejako \emph{korzystają} z tych samych zasobów informacyjnych - choć ich cele są diametralnie różne. Przykładowo:
\begin{itemize}
    \item \textbf{Agencje marketingowe} dążą do poprawy skuteczności kampanii, dokonując analizy zachowań konsumentów.
    \item \textbf{Cyberprzestępcy} wykorzystują zbliżone narzędzia (np. narzędzia do śledzenia ruchu) do przygotowania \emph{spreparowanych} komunikatów.
    \item \textbf{Brokerzy danych} dostarczają materiały i firmom, i osobom wykorzystującym dane w mniej etyczny sposób (o ile nie istnieją przepisy lub mechanizmy wewnętrzne ograniczające zbywanie informacji do pewnych sektorów).
\end{itemize}
Wspólnym mianownikiem jest tu \emph{wartość} płynąca z wiedzy o indywidualnych preferencjach i zachowaniach internautów.

\subsection{Rozwijające się platformy i nowe zastosowania}
Dodatkowym czynnikiem jest \textbf{rozszerzanie} platform interaktywnych, takich jak:
\begin{itemize}
    \item \emph{Media społecznościowe nowej generacji} (z elementami rozszerzonej rzeczywistości),
    \item \emph{Serwisy streamingowe}, w których gromadzi się szczegółowe dane o gustach i aktywnościach użytkowników (np. co, kiedy i jak często oglądają),
    \item \emph{Aplikacje do komunikacji} integrujące chatboty i analizę tekstu w czasie rzeczywistym.
\end{itemize}
Im więcej czasu internauci spędzają w takich środowiskach, tym bardziej \emph{rosną} zbiory danych, a wraz z nimi możliwość jeszcze dokładniejszego profilowania przez przeróżne podmioty.

\vspace{1em}

\section{Podsumowanie}

Przedstawione przykłady wyraźnie pokazują, że \textbf{dane} - zwłaszcza te generowane masowo i opatrzone metadanymi - \emph{stanowią} kluczowy zasób, którego znaczenie stale rośnie. Firmy reklamowe oraz marketingowe używają ich do precyzyjnego \textbf{targetowania} treści i zwiększania sprzedaży, branża HR ocenia potencjalnych pracowników w oparciu o publicznie dostępne informacje o karierze i aktywności w sieci, zaś cyberprzestępcy nie ustają w wysiłkach, by wykradać dane i \textbf{monetyzować} je poprzez phishing czy handel skradzionymi tożsamościami.

Rola brokerów danych dodatkowo \emph{zagęszcza} krajobraz, gdyż to oni scalają rozproszone fragmenty informacji, tworząc \emph{obszerne} i \emph{wszechstronne} profile użytkowników. W obliczu rosnącej liczby internautów - jak dokumentują raporty \emph{DataReportal} - rośnie także dostępność i różnorodność danych, które mogą krążyć w obiegu. Niektóre z tych praktyk leżą na granicy zgodności z oczekiwaniami społecznymi, inne naruszają prawo, zaś jeszcze inne służą zupełnie legalnym i potrzebnym analizom.

W kontekście \emph{wykorzystania danych} trudno dzisiaj mówić o jasnym podziale na „dobre” i „złe” cele, ponieważ te same informacje bywają używane zarówno do kreowania wygodniejszych usług (np. rekomendacje zakupowe), jak i do przeprowadzania skuteczniejszych ataków na użytkowników. Wartym podkreślenia pozostaje jednak fakt, że \textbf{coraz więcej} stron jest zainteresowanych pozyskiwaniem, analizowaniem i \emph{udostępnianiem} danych - co potwierdzają zarówno \emph{ENISA Threat Landscape 2022}, \emph{CERT Polska 2023}, \emph{APWG Trends Report Q4 2022}, jak i liczne publikacje \emph{EFF} oraz \emph{DataReportal}.
