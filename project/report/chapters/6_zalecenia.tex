\chapter{ZALECENIA DOTYCZĄCE PRYWATNOŚCI}
Poniżej przedstawione zostały aspekty, których przestrzeganie może nas uchronić przed zagrożeniami wynikającymi z udostepniania danych w internecie. Już na wstępie należy dodać, że eksperci jednogłośnie podkreślają, że nasza prywatność w sieci zależy przede wszystkim od naszej świadomości i ostrożności\cite{czegoNieUdostepniac}.

\section{Czego nie udostępniać w internecie?}
\subsection{Adres e-mail i numer telefonu}
Aby uniknąć spamu, phishingu, ataków socjotechnicznych, przejęcia kont, używaj oddzielnego e-maila do rejestracji i nie udostępniaj numeru telefonu publicznie.

\subsection{Adres domowy i lokalizacja geograficzna}
Aby zmniejszyć ryzyko włamania, śledzenia, identyfikacji miejsca zamieszkania, wyłącz geolokalizację w aplikacjach i unikaj publikowania informacji o podróżach czy codziennej rutynie.

\subsection{Zdjęcia nieletnich}
Dokładnie przemyśl publikację zdjęć dzieci i  ustaw ścisłe ograniczenia prywatności, aby uniknąć wykorzystanie zdjęć przez osoby nieuprawnione, ryzyko ich rozpowszechnienia bez zgody.

\subsection{Kompromitujące zdjęcia}
Nigdy nie udostępniaj intymnych zdjęć, nawet zaufanym osobom — tracisz nad nimi kontrolę. Może to doprowadzić do szantażu zdjęciami i cyberprzemocy.

\subsection{Dokumenty osobiste}
Nie publikuj dokumentów tożsamości, umów czy informacji bankowych; jeśli musisz przechowywać je online — szyfruj dane, aby uniknąć kradzieży tożsamości, oszustw finansowych.

\subsection{Opinia, skargi i kontrowersyjne komentarze}
Przemyśl każdą publikację; internet nie zapomina, a wypowiedzi mogą być błędnie zinterpretowane i spowodować  naruszenie reputacji, konflikty, nękanie.

\subsection{Prywatne rozmowy}
Korzystaj z szyfrowanych komunikatorów; nie przesyłaj wrażliwych danych w czatach, aby uniknąć ujawnienia poufnych informacji, ryzyka naruszenia prywatności innych osób.

\section{Jak się chronić?}
W wielu przypadkach nie ma jednak wyjścia. Wymagane jest udostepnienie w swoich danych w internecie. Poniżej znajduje się podsumowanie tego co warto zrobić, aby ochronić dane już przez nas udostępnione \cite{PROTECTION}.

\subsection{Aktualizuj posiadane oprogramowanie}
\begin{itemize}
    \item Przestępcy wykorzystują luki w oprogramowaniu, zanim zostaną one załatane przez producentów.
    \item Zaleca się jak najszybsze instalowanie dostępnych aktualizacji.
    \item Najlepiej włączyć automatyczne aktualizacje dla:
    \begin{itemize}
        \item oprogramowania antywirusowego,
        \item przeglądarki internetowej,
        \item systemu operacyjnego,
        \item aplikacji mobilnych.
    \end{itemize}
\end{itemize}

\subsection{Zabezpiecz domową sieć Wi-Fi}
\begin{itemize}
    \item Router jest punktem dostępowym do Internetu dla wszystkich urządzeń domowych.
    \item Zainfekowane urządzenie może rozprzestrzenić złośliwe oprogramowanie na inne w sieci.
    \item Zaleca się zabezpieczenie routera hasłem i aktualizacją oprogramowania.
\end{itemize}

\subsection{Skonfiguruj ustawienia prywatności}
Na swoich kontach - ogranicz widoczność treści tylko do osób zaufanych.

\subsection{Używaj silnych, unikalnych haseł}
Nieopartych na łatwo dostępnych danych osobowych. Regularnie je zmieniaj i nie używaj ich wielokrotnie.
\begin{itemize}
    \item Hasło powinno być długie (co najmniej 15 znaków) i zawierać duże i małe litery, cyfry oraz znaki specjalne.
    \item Można stosować losowe hasła lub frazy z przypadkowych słów.
    \item Alternatywy:
    \begin{itemize}
        \item Automatycznie generowane hasła przez przeglądarkę.
        \item Menedżery haseł - umożliwiają tworzenie i bezpieczne przechowywanie haseł. Dostępne w większości przeglądarek. Dostępne są również dedykowane aplikacje przechowujące hasła.
    \end{itemize}
\end{itemize}

\subsection{Korzystaj z uwierzytelniania dwuskładnikowego (2FA)}
\begin{itemize}
    \item Nawet silne hasło można złamać — 2FA zwiększa bezpieczeństwo.
    \item Najczęściej spotykane formy:
    \begin{itemize}
        \item jednorazowe kody przez SMS lub e-mail,
        \item aplikacje autoryzujące,
        \item fizyczne klucze bezpieczeństwa.
    \end{itemize}
\end{itemize}

\subsection{Uważaj na podejrzane linki}
Sprawdzaj źródło wiadomości i nie klikaj w nieznane odnośniki. Przestępcy stosują wiadomości e-mail lub SMS-y z fałszywymi linkami i załącznikami.

\subsection{Korzystaj z zaufanych źródeł cyberbezpieczeństwa}
Np. Incibe, OSI.

\subsection{Nie publikuj wrażliwych informacji}
Szczególnie tych dotyczących lokalizacji, planów i życia prywatnego.

\subsection{Wyłącz geolokalizację}
Jeśli nie jest potrzebna.

\subsection{Zachowaj ostrożność w komentarzach}
Nawet pozorna anonimowość może być złudna.

\subsection{Rozważ korzystanie z usług typu VPN}
Szczególnie w publicznych sieciach Wi-Fi (na laptopie, smartfonie, tablecie).

\subsection{Unikaj podawania numeru telefonu}
Tam, gdzie nie jest to absolutnie konieczne, ewentualnie do celów bezpieczeństwa (np. 2FA).

\subsection{Świadomie analizuj, z jakich usług korzystasz i jakimi danymi płacisz za „darmowy” dostęp}

\subsection{Zastrzeż swój numer PESEL}
Od niedawna w Polsce można zastrzec swój numer PESEL, aby nikt nie wykorzystał go bez wiedzy osoby, do której ten numer należy.

Z zastrzeżonym numerem PESEL można:
\begin{itemize}
    \item zarejestrować się do lekarza,
    \item zrealizować receptę,
    \item wypłacić środki w bankomacie,
    \item zlecić przelew bankowy,
    \item wyjechać za granicę lub
    \item załatwić sprawę urzędową.
\end{itemize}

Natomiast z zastrzeżonym numerem PESEL:
\begin{itemize}
    \item nie można zaciągnąć kredytu, pożyczki, leasingu,
    \item nie można otworzyć nowego rachunku bankowego,
    \item nie można zmienić umowy kredytu lub pożyczki,
    \item nie można wypłacić w placówce banku więcej pieniędzy niż trzykrotność minimalnego wynagrodzenia,
    \item nie można załatwić niektórych spraw notarialnych,
    \item nie można otrzymać duplikatu karty SIM.
\end{itemize}

Należy cofnąć zastrzeżenie numeru PESEL w momencie, gdy zajdzie potrzeba wykonania powyższych czynności.\\
Od 1 czerwca 2024 r. instytucje finansowe (np. banki) mają obowiązek weryfikować, czy numer PESEL jest zastrzeżony przy zawieraniu np. umowy kredytu lub pożyczki.\\
Funkcjonalność ta oferowana przez urząd państwowy pozwala uniknąć wielu przykrych i niebezpiecznych sytuacji. Zastrzeżenie PESELu to sprawa, którą można załatwić bardzo szybko - przez internet, bądź osobiście w Urzędzie \cite{pesel}.

\section{Co zrobić jeśli wyciekną dane osobowe?}

Wyciek danych osobowych to incydent, w wyniku którego dochodzi do niezamierzonego lub celowego ujawnienia danych identyfikujących daną osobę. Może on mieć miejsce zarówno w przestrzeni cyfrowej (np. Internet), jak i offline, np. przez zgubienie lub kradzież dokumentów tożsamości \cite{WYCIEK}.

\subsection{Przyczyny wycieków danych}
\begin{itemize}
    \item ataki hakerskie lub działania przestępcze,
    \item błędy użytkowników,
    \item awarie systemów informatycznych,
    \item niesprawny sprzęt komputerowy.
\end{itemize}

\subsection{Możliwe skutki wycieku danych}
\begin{itemize}
    \item kradzież tożsamości (wyłudzenia kredytów, zawieranie umów na cudze dane),
    \item utrata kontroli nad swoimi informacjami,
    \item straty finansowe i problemy z prawem,
    \item naruszenie reputacji i dobrego imienia,
    \item szkody gospodarcze i społeczne,
    \item negatywny wpływ na środowisko zawodowe.
\end{itemize}

\subsection{Jak sprawdzić, czy doszło do wycieku danych?}

Nie zawsze otrzymujemy bezpośrednie powiadomienie o naruszeniu danych, dlatego warto monitorować swoją aktywność w sieci. Pomocne w tym są następujące narzędzia:

\begin{itemize}
    \item \textbf{\href{https://haveibeenpwned.com}{haveibeenpwned.com}} - pozwala sprawdzić, czy dane logowania (adres e-mail, hasło) znajdują się w bazie znanych wycieków.
    \item \textbf{\href{https://www.bik.pl/}{Biuro Informacji Kredytowej (BIK)}} - umożliwia monitorowanie zapytań kredytowych powiązanych z Twoimi danymi. W razie próby ich użycia, system wyśle powiadomienie.
    \item \textbf{\href{https://chronpesel.pl/}{Lokalizator wycieku danych}} - pomocne narzędzie weryfikujące, czy PESEL lub inne dane zostały ujawnione.
\end{itemize}

\subsection{Co zrobić po wycieku danych?}

W przypadku potwierdzenia lub podejrzenia wycieku danych osobowych, należy działać natychmiast. Poniżej przedstawiono najważniejsze kroki:

\subsubsection{Wyloguj się i zmień hasła}
\begin{itemize}
    \item Wyloguj się ze wszystkich kont, na których mogły zostać wykorzystane wyciekłe dane.
    \item Zmień hasła na silne i unikalne - co najmniej 12 znaków, zawierające litery, cyfry i znaki specjalne.
    \item Stosuj zasadę: jedno konto = jedno hasło.
\end{itemize}

\subsubsection{Zastrzeż dokumenty w banku}
\begin{itemize}
    \item W przypadku wycieku danych dowodu osobistego, karty płatniczej itp. natychmiast je zastrzeż.
    \item Można to zrobić w dowolnym banku uczestniczącym w programie „Dokumenty Zastrzeżone”.
    \item W przypadku kart płatniczych można skorzystać z \textbf{Systemu Zastrzegania Kart}.
\end{itemize}

\subsubsection{Zgłoś sprawę na policję}
\begin{itemize}
    \item Jeżeli uważasz, że doszło do przestępstwa, zgłoś to na policję.
    \item Zachowaj potwierdzenie zgłoszenia - może okazać się dowodem w postępowaniu sądowym lub administracyjnym.
\end{itemize}

Bezpieczeństwo w sieci dobrze podsumowuje poniższy cytat:
\begin{center}
\textit{Twoje dane to Twoja odpowiedzialność.\\
W świecie cyfrowym prywatność nie jest dana raz na zawsze - trzeba ją stale chronić.}
\end{center}

\chapter{CZY ISTNIEJĄ POZYTYWNE ASPEKTY UDOSTĘPNIANIA DANYCH?}

Choć dominująca narracja dotycząca udostępniania danych w Internecie koncentruje się na ryzykach i zagrożeniach, warto podkreślić, że istnieją także wymierne i konstruktywne korzyści płynące z kontrolowanego i świadomego dzielenia się informacjami. Poniżej przedstawiono wybrane obszary, w których dane osobowe - przy właściwym zarządzaniu - mogą wspierać rozwój społeczny, technologiczny i indywidualny.\cite{STRATEGIE}

\section{Lepszy dostęp do usług cyfrowych}

Jednym z głównych beneficjentów udostępniania danych są usługi internetowe, które dzięki analizie informacji o użytkownikach mogą lepiej odpowiadać na ich potrzeby. Przykładem są personalizowane rekomendacje w sklepach internetowych, inteligentne asystenty, czy zoptymalizowane systemy pomocy technicznej. W efekcie użytkownicy otrzymują szybsze, trafniejsze i bardziej dopasowane rozwiązania.

\section{Wsparcie dla nauki i innowacji}

W środowiskach badawczych i technologicznych dane mają fundamentalne znaczenie. Otwarte zasoby informacyjne umożliwiają wymianę wiedzy pomiędzy ośrodkami naukowymi, przyspieszając procesy badawcze oraz rozwój nowoczesnych technologii - np. w obszarze sztucznej inteligencji, medycyny spersonalizowanej czy urbanistyki. Udostępnianie danych w modelach typu open data staje się podstawą innowacyjnych rozwiązań społecznych i gospodarczych.

\section{Zarządzanie przestrzenią i politykami społecznymi}

Dane gromadzone przez użytkowników, np. dotyczące mobilności, korzystania z usług publicznych czy preferencji konsumenckich, stanowią cenne źródło wiedzy dla miast i instytucji publicznych. Pozwalają one optymalizować infrastrukturę, planować transport zbiorowy, przewidywać potrzeby lokalnych społeczności czy usprawniać zarządzanie kryzysowe.

\section{Budowanie tożsamości cyfrowej}

Dla użytkowników indywidualnych, szczególnie przedstawicieli młodego pokolenia, dzielenie się informacjami w sieci stanowi element wyrażania siebie. Tworzenie i kształtowanie własnej tożsamości cyfrowej umożliwia budowanie relacji społecznych, uczestnictwo w kulturze internetowej oraz rozwój kompetencji komunikacyjnych i medialnych.\cite{CENTRUM}

\section{Warunki korzystnego udostępniania danych}

Należy jednak podkreślić, że powyższe korzyści są osiągalne jedynie wtedy, gdy udostępnianie danych odbywa się świadomie i przy zachowaniu zasad prywatności. Kluczowe znaczenie ma znajomość ustawień prywatności, zrozumienie mechanizmów przetwarzania danych przez platformy internetowe oraz umiejętność selekcji informacji udostępnianych publicznie. Tylko w takich warunkach możliwe jest bezpieczne czerpanie korzyści z obecności w przestrzeni cyfrowej.
