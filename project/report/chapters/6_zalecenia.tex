\chapter{ZALECENIA DOTYCZĄCE PRYWATNOŚCI}
\section{Jak się chronić? \cite{czegoNieUdostepniac}}
Nadmierne lub nieświadome udostępnianie danych może narazić nas na poważne zagrożenia, takie jak kradzież tożsamości, oszustwa czy cybernękanie.

\subsection{Adres e-mail i numer telefonu}
\textbf{Zagrożenia:} spam, phishing, ataki socjotechniczne, przejęcie kont.\\
\textbf{Rekomendacja:} używaj oddzielnego e-maila do rejestracji; nie udostępniaj numeru telefonu publicznie.

\subsection{Adres domowy i lokalizacja geograficzna}
\textbf{Zagrożenia:} ryzyko włamania, śledzenia, identyfikacji miejsca zamieszkania.\\
\textbf{Rekomendacja:} wyłącz geolokalizację w aplikacjach; unikaj publikowania informacji o podróżach czy codziennej rutynie.

\subsection{Zdjęcia nieletnich}
\textbf{Zagrożenia:} wykorzystanie zdjęć przez osoby nieuprawnione, ryzyko ich rozpowszechnienia bez zgody.\\
\textbf{Rekomendacja:} dokładnie przemyśl publikację zdjęć dzieci; ustaw ścisłe ograniczenia prywatności.

\subsection{Kompromitujące zdjęcia}
\textbf{Zagrożenia:} \textit{sextortion} (szantaż zdjęciami), cyberprzemoc.\\
\textbf{Rekomendacja:} nigdy nie udostępniaj intymnych zdjęć, nawet zaufanym osobom — tracisz nad nimi kontrolę.

\subsection{Dokumenty osobiste}
\textbf{Zagrożenia:} kradzież tożsamości, oszustwa finansowe.\\
\textbf{Rekomendacja:} nie publikuj dokumentów tożsamości, umów czy informacji bankowych; jeśli musisz przechowywać je online — szyfruj dane.

\subsection{Opinia, skargi i kontrowersyjne komentarze}
\textbf{Zagrożenia:} naruszenie reputacji, konflikty, nękanie.\\
\textbf{Rekomendacja:} przemyśl każdą publikację; internet nie zapomina, a wypowiedzi mogą być błędnie zinterpretowane.

\subsection{Prywatne rozmowy}
\textbf{Zagrożenia:} ujawnienie poufnych informacji, ryzyko naruszenia prywatności innych osób.\\
\textbf{Rekomendacja:} korzystaj z szyfrowanych komunikatorów; nie przesyłaj wrażliwych danych w czatach.

\section{Bezpieczny dostęp do internetu i mediów społecznościowych}

\begin{itemize}
  \item \textbf{Skonfiguruj ustawienia prywatności} na swoich kontach - ogranicz widoczność treści tylko do osób zaufanych.
  \item \textbf{Używaj silnych, unikalnych haseł} - regularnie je zmieniaj i nie używaj ich wielokrotnie.
  \item \textbf{Uważaj na podejrzane linki} - sprawdzaj źródło wiadomości i nie klikaj w nieznane odnośniki.
  \item \textbf{Korzystaj z zaufanych źródeł cyberbezpieczeństwa} - np. Incibe, OSI.
\end{itemize}

\begin{center}
\textit{Twoje dane to Twoja odpowiedzialność.\\
W świecie cyfrowym prywatność nie jest dana raz na zawsze - trzeba ją stale chronić.}
\end{center}