\chapter{ZALECENIA DOTYCZĄCE PRYWATNOŚCI}
Poniżej przedstawione zostały aspekty, których przestrzeganie może nas uchronić przed zagrożeniami wynikającymi z udostepniania danych w internecie. Już na wstępie należy dodać, że eksperci jednogłośnie podkreślają, że nasza prywatność w sieci zależy przede wszystkim od naszej świadomości i ostrożności\cite{czegoNieUdostepniac}.

\section{Czego nie udostępniać w internecie?}
\subsection{Adres e-mail i numer telefonu}
Aby uniknąć spamu, phishingu, ataków socjotechnicznych, przejęcia kont, używaj oddzielnego e-maila do rejestracji i nie udostępniaj numeru telefonu publicznie.

\subsection{Adres domowy i lokalizacja geograficzna}
Aby zmniejszyć ryzyko włamania, śledzenia, identyfikacji miejsca zamieszkania, wyłącz geolokalizację w aplikacjach i unikaj publikowania informacji o podróżach czy codziennej rutynie.

\subsection{Zdjęcia nieletnich}
Dokładnie przemyśl publikację zdjęć dzieci i  ustaw ścisłe ograniczenia prywatności, aby uniknąć wykorzystanie zdjęć przez osoby nieuprawnione, ryzyko ich rozpowszechnienia bez zgody.

\subsection{Kompromitujące zdjęcia}
Nigdy nie udostępniaj intymnych zdjęć, nawet zaufanym osobom — tracisz nad nimi kontrolę. Może to doprowadzić do szantażu zdjęciami i cyberprzemocy.

\subsection{Dokumenty osobiste}
Nie publikuj dokumentów tożsamości, umów czy informacji bankowych; jeśli musisz przechowywać je online — szyfruj dane, aby uniknąć kradzieży tożsamości, oszustw finansowych.

\subsection{Opinia, skargi i kontrowersyjne komentarze}
Przemyśl każdą publikację; internet nie zapomina, a wypowiedzi mogą być błędnie zinterpretowane i spowodować  naruszenie reputacji, konflikty, nękanie.

\subsection{Prywatne rozmowy}
Korzystaj z szyfrowanych komunikatorów; nie przesyłaj wrażliwych danych w czatach, aby uniknąć ujawnienia poufnych informacji, ryzyka naruszenia prywatności innych osób.

\section{Jak się chronić?}
W wielu przypadkach nie ma jednak wyjścia. Wymagane jest udostepnienie w swoich danych w internecie. Poniżej znajduje się podsumowanie tego co warto zrobić, aby ochronić dane już przez nas udostępnione.

\begin{itemize}
  \item \textbf{Skonfiguruj ustawienia prywatności} na swoich kontach - ogranicz widoczność treści tylko do osób zaufanych.
  \item \textbf{Używaj silnych, unikalnych haseł} - regularnie je zmieniaj i nie używaj ich wielokrotnie.
  \item \textbf{Uważaj na podejrzane linki} - sprawdzaj źródło wiadomości i nie klikaj w nieznane odnośniki.
  \item \textbf{Korzystaj z zaufanych źródeł cyberbezpieczeństwa} - np. Incibe, OSI,
  \item \textbf{Nie publikuj wrażliwych informacji} - szczególnie tych dotyczących lokalizacji, planów i życia prywatnego,
  \item \textbf{Wyłącz geolokalizację} jeśli nie jest potrzebna,
  \item \textbf{Używaj silnych i unikalnych haseł} nieopartych na łatwo dostępnych danych osobowych,
  \item \textbf{Zachowaj ostrożność w komentarzach} nawet pozorna anonimowość może być złudna,
  \item \textbf{Rozważ korzystanie z usług typu VPN}, szczególnie w publicznych sieciach Wi-Fi (na laptopie, smartfonie, tablecie).
  \item \textbf{Unikaj podawania numeru telefonu} tam, gdzie nie jest to absolutnie konieczne, ewentualnie do celów bezpieczeństwa (np. 2FA).
  \item \textbf{Świadomie analizuj, z jakich usług korzystasz i jakimi danymi płacisz za „darmowy” dostęp.}
  \item \textbf{Zastrzeż swój numer PESEL}. Od niedawna w Polsce można zastrzec swój numer PESEL, aby nikt nie wykorzystał go bez wiedzy osoby, do której ten numer należy. Z zastrzeżonym numerem PESEL można:
  \begin{itemize}
    \item zarejestrować się do lekarza,
    \item zrealizować receptę,
    \item wypłacić środki w bankomacie,
    \item zlecić przelew bankowy,
    \item wyjechać za granicę lub
    \item załatwić sprawę urzędową.
  \end{itemize}
  Natomiast z zastrzeżonym numerem PESEL:
  \begin{itemize}
    \item Nie można zaciągnąć kredytu, pożyczki, leasingu,
    \item Nie można otworzyć nowego rachunku bankowego,
    \item Nie można zmienić umowy kredytu lub pożyczki,
    \item Nie można wypłacić w placówce banku więcej pieniędzy niż trzykrotność minimalnego wynagrodzenia,
    \item Nie można załatwić niektórych spraw notarialnych,
    \item Nie można otrzymać duplikatu karty SIM.
\end{itemize}
Należy cofnąć zastrzeżenie numeru PESEL w momencie gdy zajdzie potrzeba wykonania powyższych czynności.\\\
Od 1 czerwca 2024 r. instytucje finansowe (np. banki) mają obowiązek weryfikować, czy numer PESEL jest zastrzeżony przy zawieraniu np. umowy kredytu lub pożyczki. \\
Funkcjonalność ta oferowana przez urząd państwowy pozwala uniknąć wiele przykrych i niebezpiecznych sytuacji. Zastrzeżenie PESELu to sprawa, którą można załatwić bardzo szybko - przez internet, bądź osobiście w Urzędzie\cite{pesel}.
\end{itemize}

Bezpieczeństwo w sieci dobrze podsumowuje poniższy cytat:
\begin{center}
\textit{Twoje dane to Twoja odpowiedzialność.\\
W świecie cyfrowym prywatność nie jest dana raz na zawsze - trzeba ją stale chronić.}
\end{center}