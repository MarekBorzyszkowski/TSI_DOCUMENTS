\chapter{ŹRÓDŁA POZYSKIWANIA INFORMACJI}
\enlargethispage{20pt}

\section{Wprowadzenie - co mówi prawo?}

Informacje o osobach mogą być przechowywane w publicznych źródłach, takich jak Internet, media społecznościowe czy rejestry publiczne. Często sami użytkownicy publikują te dane. Jednak fakt, że dane są publiczne, nie oznacza, że można je wykorzystywać w dowolny sposób.

Umieszczenie danych osobowych w otwartych źródłach, np. na publicznych stronach internetowych, jest uznawane za przetwarzanie danych osobowych. Oznacza to, że zastosowanie ma Rozporządzenie o Ochronie Danych Osobowych (RODO), a więc konieczne jest posiadanie podstawy prawnej do takiego przetwarzania.

Wyjątkiem są cele osobiste i domowe. RODO nie obowiązuje, jeśli dane są przetwarzane wyłącznie w celach osobistych lub domowych. Można umieścić własne dane osobowe w otwartych źródłach - nie trzeba mieć ku temu podstawy prawnej. Każda osoba może dowolnie dysponować własnymi danymi osobowymi.

W przypadku zamiaru umieszczenia danych osobowych innych osób w otwartych źródłach:
\begin{itemize}
    \item należy posiadać podstawę prawną przetwarzania danych,
    \item należy posiadać konkretny i uzasadniony cel,
    \item należy spełniać pozostałe wymagania RODO.
\end{itemize}

Obowiązki te należy spełnić \textbf{przed} opublikowaniem danych, aby uniknąć naruszenia przepisów RODO.

Dodatkowo obowiązuje zakaz przetwarzania danych szczególnej kategorii (np. dane zdrowotne, poglądy polityczne) oraz danych karnych. Można je przetwarzać jedynie wtedy, gdy zastosowanie znajduje wyjątek od tego zakazu. Publikując dane, należy samodzielnie ocenić, czy:
\begin{itemize}
    \item dane należą do kategorii szczególnych lub karnych,
    \item można powołać się na wyjątek od zakazu ich przetwarzania.
\end{itemize}

Dane osobowe, które zgodnie z RODO zostały legalnie opublikowane w otwartym źródle, można wykorzystywać wyłącznie do użytku osobistego lub domowego (np. prywatny kalendarz urodzin, książka adresowa). W takich przypadkach RODO nie ma zastosowania.

\textbf{Niedozwolone} jest wykorzystywanie tych danych w celach zawodowych lub komercyjnych, jak również ich udostępnianie poza najbliższym gronem (np. rodziną, przyjaciółmi).

Dane mogą być legalnie dostępne w otwartym źródle, ale to nie oznacza, że wolno je automatycznie przetwarzać ponownie.

\textbf{Zgoda} osoby, której dane dotyczą, jest często niewystarczająca jako podstawa do ponownego przetwarzania — zwłaszcza w przypadku przetwarzania na większą skalę (np. gromadzenie danych do baz).

Każde nowe przetwarzanie danych wymaga sprawdzenia, czy konieczne jest przeprowadzenie \textbf{Oceny Skutków dla Ochrony Danych} (ang. DPIA - Data Protection Impact Assessment).

DPIA pozwala:
\begin{itemize}
    \item zidentyfikować ryzyka dla prywatności,
    \item podjąć środki zaradcze jeszcze przed rozpoczęciem przetwarzania.
\end{itemize}

Nawet jeśli DPIA nie jest obowiązkowa, jej przeprowadzenie jest zalecane.

Organizacje, które automatycznie przeszukują Internet w celu gromadzenia danych osobowych, również są podmiotami przetwarzającymi dane i muszą przestrzegać przepisów RODO. Oznacza to, że muszą:
\begin{itemize}
    \item wykazać istnienie podstawy prawnej,
    \item stosować zasadę minimalizacji danych,
    \item przetwarzać tylko dane niezbędne do realizacji określonego celu.
\end{itemize}

Dla danych szczególnych i karnych obowiązuje zakaz przetwarzania. Nawet jeśli dane te zostały legalnie opublikowane, nie można ich ponownie przetwarzać — \textbf{chyba, że} poza podstawą prawną istnieje również uzasadniony wyjątek od zakazu przetwarzania.

To, że dane są publicznie dostępne (lub zostały opublikowane przez samą osobę), nie oznacza, że można je bez ograniczeń ponownie wykorzystywać. Każde nowe przetwarzanie danych wymaga osobnej analizy zgodności z przepisami RODO \cite{SOURCES}.

Chociaż wszelkie źródła informacji muszą być prowadzone z uwzględnieniem praw osób, których dane dotyczą i w granicach wyznaczonych przez obowiązujące przepisy prawa, to w praktyce użytkownik jest i tak narażony na szereg zagrożeń.

\section{Identyfikacja i analiza publicznie dostępnych źródeł}

W dobie powszechnego dostępu do Internetu oraz ogromnej popularności mediów społecznościowych, zdobycie informacji na temat osoby prywatnej, przedsiębiorcy czy pracownika jest dziś łatwiejsze niż kiedykolwiek wcześniej. Użytkownicy często sami - świadomie lub nieświadomie - pozostawiają po sobie szereg danych, które można wykorzystać do stworzenia precyzyjnego profilu.

Według raportu „Digital 2021” (We Are Social i Hootsuite), z Internetu w Polsce korzysta 31,97 mln osób, a 25,9 mln aktywnie używa mediów społecznościowych. To oznacza, że większość społeczeństwa codziennie generuje cyfrowe ślady, które mogą być później analizowane.\cite{zrodlo}

\subsection{Media społecznościowe}

Facebook, Instagram, X, LinkedIn czy Snapchat gromadzą dane zarówno świadomie podawane przez użytkowników (np. miejsce pracy), jak i te zbierane automatycznie - lokalizacja, adres IP, urządzenia.
Dane mogą pochodzić także z aplikacji firm trzecich, programów lojalnościowych czy partnerów marketingowych.\cite{zrodlo2}

Szczegółowe zestawienie, jakie dane są gromadzone w mediach społecznościowych zostały przedstawione w poniższej tabeli. \cite{zrodloArtykul}
\begin{longtable}{p{8.5cm}|c|c|c}
\toprule
\textbf{Zakres danych / Działanie} & \textbf{Facebook} & \textbf{X} & \textbf{LinkedIn} \\
\midrule
\multicolumn{4}{l}{\textbf{DANE OSOBOWE}} \\
Imię i nazwisko & + & + & + \\
E-mail & + & + & + \\
Numer telefonu & + & + & + \\
Data urodzenia & + & + & + \\
Adres zamieszkania & + & + & + \\
Poprzednie miejsca zamieszkania & + &  &  \\
Zdjęcia profilowe & + & + & + \\
Rodzina i związki & + &  &  \\
Wykształcenie & + &  & + \\
Języki obce & + &  & + \\
Poglądy polityczne & + &  &  \\
Przekonania religijne & + &  &  \\
Wydarzenia z życia & + &  &  \\
\midrule
\multicolumn{4}{l}{\textbf{DANE ZAWODOWE I LOKALIZACYJNE}} \\
Miejsce pracy & + &  & + \\
Kwalifikacje & + &  & + \\
Wynagrodzenie &  &  & + \\
Płatności & + &  & + \\
Lokalizacja & + & + & + \\
Urządzenia & + & + & + \\
Adres IP & + & + & + \\
Kalendarz &  &  & + \\
\midrule
\multicolumn{4}{l}{\textbf{ŹRÓDŁA POZYSKIWANIA INFORMACJI}} \\
Od użytkownika & + & + & + \\
Od innych użytkowników & + & + & + \\
Od partnerów zewnętrznych & + &  & + \\
Z aplikacji zewnętrznych & + & + &  \\
Z aktywności (kliknięcia) & + & + &  \\
\midrule
\multicolumn{4}{l}{\textbf{ZASTOSOWANIA I UDOSTĘPNIANIE}} \\
Reklamy & + & + & + \\
Personalizacja & + & + & + \\
Oznaczanie na zdjęciach & + &  &  \\
Meldowanie w lokalizacjach & + & + &  \\
Sugestie kontaktów & + &  & + \\
Udostępnianie firmom & + & + & + \\
Cele badawcze i analityczne & + & + & + \\
\bottomrule
\end{longtable}

Jesli chodzi o Facebook, to gromadzi on najszerszy zakres danych - zarówno prywatnych, jak i zawodowych, w tym dane kontaktowe, lokalizacyjne, relacyjne i behawioralne. Dane są również pozyskiwane od partnerów i aplikacji zewnętrznych. Warto zaznaczyć, że domyślne ustawienia prywatności, których większość użytkowników nigdy dokładnie nie czyta przed akceptacją, bardzo często sprzyjają upublicznieniu informacji, a około 40\% użytkowników nie zmienia nigdy ustawień prywatności, co czyni ich dane łatwo dostępnymi.

W aplikacji X domyślnie wszystkie posty (tweety) są publiczne. Dodatkowo dane gromadzone również poprzez partnerów i aplikacje zewnętrzne, a polityka firmy jest dosyć liberalna i udostępnia dane badaczom.

Firma LinkedIn dodatkowo profiluje użytkowników głównie pod kątem danych zawodowych (miejsce pracy, kwalifikacje, języki), które potem są wykorzystywane komercyjnie i sprzedawane firmom rekrutacyjnym. Może być to szczególnie niebezpieczne przy atakach typu spear-phishing, gdzie precyzyjne informacje są używane do podszywania się i oszustw.

\subsection{Fora, sekcje komentarzy, blogi}

Uznawane za „anonimowe”, w rzeczywistości łatwe do deanonimizacji. Powtarzające się pseudonimy, zdjęcia profilowe, adresy IP - wszystko to pozwala powiązać konto z konkretną osobą, po mniej lub bardziej wnikliwym poszukiwaniu.

\subsection{Rejestry publiczne i portale firmowe}
Z racji powszechnej cyfryzacji, przed udostępnieniem niektórych danych w internecie nie można się uchronić. Szczególnie osoby prowadzące firmy lub działalności gospodarcze, a nawet osoby posiadające nieruchomości. 
Nie wspominając o tym, że prawie każdy dorosły człowiek w dzisiejszych czasach pracuje w jakiejś firmie, gdzie często dane kontaktowe, adresy e-mail, a czasem także numery telefonów pracowników czy właścicieli firm są udostępniane na stronie firm, bardzo często nawet bez wiedzy samych zainteresowanych.\cite{zrodloDzialalnosc} \\
Niektóre dane są dostępne w postaci rejestrów internetowych:

\begin{itemize}
  \item \textbf{CEIDG} - zawiera dane o jednoosobowej działalności gospodarczej, w tym adres, który nierzadko jest miejscem zamieszkania. 
  \item \textbf{KRS} - dane zarządu, często z numerem PESEL, umożliwia pozyskanie numeru PESEL i daty urodzenia członków zarządu spółek.
  \item \textbf{CRBR} - udostępnia dane beneficjentów rzeczywistych, w tym numery PESEL udziałowców spółek.
  \item \textbf{Elektroniczne Księgi Wieczyste} - umożliwiają dotarcie do danych właścicieli nieruchomości.
  \item \textbf{Portale ogłoszeniowe, firmowe} - e-maile, telefony, nazwiska
\end{itemize}

\subsection{Urządzenia IoT}

Inteligentne urządzenia domowe (głośniki, żarówki, lodówki) zbierają dane o rutynie, lokalizacji i obecności użytkownika. Coraz więcej urządzeń domowych (np. tostery, żarówki, głośniki) jest wyposażanych w moduły internetowe. Nawet jeśli nie oferują realnych funkcji online, dane z nich są zbierane i mogą być wykorzystywane marketingowo. Koszt wdrożenia łączności IoT dla producenta jest niewielki, a dane mają wysoką wartość.

\begin{quote}
„Kupując sprzęt domowy, możemy nieświadomie nabyć urządzenie IoT. Kluczowym zasobem są dane - które mogą zostać sprzedane lub wykorzystane przez producenta.” - Mikko Hypponen
\end{quote}

\subsection{Czat AI jako nieświadome źródło informacji}
Coraz więcej osób korzysta z narzędzi opartych na sztucznej inteligencji (AI), takich jak ChatGPT czy Bing Chat, aby wspierać się w pracy zawodowej. Według badań aż 43\% pracowników używa AI do realizacji swoich zadań.\cite{ai} Choć AI oferuje wygodne i szybkie odpowiedzi, istnieją zagrożenia dotyczące prywatności i bezpieczeństwa danych, o których warto pamiętać, że:

\begin{itemize}
  \item dane mogą być przetwarzane lub analizowane przez ludzi,
  \item platformy zastrzegają prawo do przechowywania zapytań,
  \item AI modele, takie jak ChatGPT, mogą uczyć się na podstawie zadawanych pytań. Dlatego nie powinno się wprowadzać do nich żadnych danych osobowych ani poufnych.
\end{itemize}

\textbf{Co zdarza się ludziom udostępnić, a nie powinno się tego robić:}

\begin{itemize}
  \item Imię i nazwisko,
  \item Data urodzenia,
  \item Adres zamieszkania,
  \item Numer dowodu osobistego,
  \item Telefon, adres e-mail,
  \item Kod źródłowy lub dane z projektów w fazie rozwoju,
  \item Poufne informacje o produktach i usługach,
  \item Niezaprezentowane jeszcze pomysły lub rozwiązania,
  \item Numery kart kredytowych,
  \item Numery kont bankowych,
  \item Hasła i dane logowania,
  \item Wyniki badań,
  \item Diagnozy lekarskie,
  \item Informacje dotyczące leczenia.
\end{itemize}

\textbf{Aby korzystać z AI w sposób odpowiedzialny i bezpieczny, warto stosować się do kilku prostych zasad:}

\begin{itemize}
  \item Używanie jedynie informacji ogólnodostępnych i nieidentyfikujących,
  \item Podawanie dane fikcyjne lub zanonimizowane, jeśli konieczne jest podanie przykładu.
  \item Nie publikowanie szczegółowych planów podróży lub wakacji — może to narażać na kradzież lub inne ryzyko offline.
  \item Założenie drugiego, anonimowego konta, które nie jest powiązane z właściwą tożsamością.
  \item Korzystanie z bezpiecznego połączenia internetowego — unikanie otwartych sieci Wi-Fi w kawiarniach czy hotelach.
  \item Nie udostępnianie haseł ani nie proszenie AI o ich generowanie na potrzeby konkretnej usługi.
  \item Zachowanie ostrożności przy danych zdrowotnych, finansowych i prywatnych,
  \item Zapoznanie się z regulaminem i polityką prywatności chatu AI,
  \item Sprawdzenie, jakie dane są przechowywane, jak długo i komu mogą być udostępniane,
  \item Zwrócenie uwagi, czy narzędzie przewiduje możliwość przeglądu zapytań przez człowieka w celach jakościowych.
\end{itemize}

\section{W jaki sposób dane trafiają w niepowołane ręce?}
Chociaż skoro myślimy, że udostepniamy bardzo mało danych, bądź wcale, istnieją aktualnie techniki, które są na tyle zaawansowane, że na podstawie jednej, bądź kilku informacji można stworzyć czyiś profil osobowy. Do jednej z nich zalicza się biały wywiad. 
Open Source Intelligence, cyli biały wywiad, to wyspecjalizowana technika polegająca na analizie dostępnych publicznie informacji w celu identyfikacji i gromadzenia danych o wybranej osobie.

\begin{itemize}
  \item przeszukiwanie stron internetowych pod kątem danych kontaktowych,
  \item wyszukiwanie kont społecznościowych na podstawie adresu e-mail lub loginu,
  \item analizę zdjęć w celu rozpoznania wizerunku i odnalezienia powiązanych treści,
  \item korelowanie fragmentarycznych danych z różnych źródeł (np. fora, blogi, komentarze).
\end{itemize}

Aktualne narzędzie umożliwiają bardzo wnikliwą analizę, między innymi:

\begin{itemize}
  \item automatyczne przeszukiwanie domen internetowych w celu odnalezienia adresów e-mail,
  \item sprawdzanie, na jakich portalach społecznościowych zarejestrowano konkretne nazwy użytkowników,
  \item analizę zdjęć za pomocą sztucznej inteligencji w celu identyfikacji osoby.
\end{itemize}

Chociaż prawda jest taka, że nie potrzeba żadnych profesjonalnych narzędzi, aby komuś uprzykrzyć życie. Dostęp do danych wrażliwych może mieć każda osoba prywatna, wystarczy tylko odrobina negatywnej motywacji oraz chwila czasu spędzona na poszukiwaniu i wnioskowaniu, ale o tym wiecej w nastepnych rozdziałach. 